\newif\ifIntro
%\Introtrue
\newif\ifLabtwo
%\Labtwotrue
\newif\ifLabthree
%\Labthreetrue
\newif\ifLabfour
%\Labfourtrue
\newif\ifLabfive
%\Labfivetrue
\newif\ifExample
\Exampletrue


\documentclass[a4paper,11pt,twoside]{article}
\usepackage{a4wide}
\usepackage{amssymb,amsmath}
\usepackage{palatino,mathpazo}
\usepackage{exercise}
\usepackage{comment}
\usepackage{url}
\usepackage{graphicx}
\usepackage[medium,compact]{titlesec}  
\usepackage{tabularx}
\usepackage[dutch]{babel}
\usepackage{xspace}
%\usepackage{misc}
\usepackage{enumitem}
\usepackage[usenames]{color}
\input{dvipsnam.def}
\usepackage{colortbl}
\usepackage[final]{pdfpages}
\usepackage{bmc_defs}
\usepackage{verbatim}
\setcounter{secnumdepth}{3}
\setlength{\parindent}{0.0cm}
\setlength{\parskip}{0.1cm}

%\title{\textbf{\huge Practicals Introduction to Biomedical Computing}}
%\author{Jos Roerdink}
%\date{2013}
  
\renewcommand{\ExerciseHeader}{{%
\textbf{\large\ExerciseHeaderDifficulty\ExerciseName\ %
\ExerciseHeaderNB\ExerciseHeaderTitle\ExerciseHeaderOrigin}}\medskip}
\def\ExerciseName{Assignment}
\renewcommand{\theenumi}{\alph{enumi}}



\newcounter{lab}[section]
\newcommand{\lab}[2]{{\refstepcounter{lab}\label{#1}\vspace{2ex}\noindent\bf\large
  Practicum \ref{#1}.\ {#2} }\bigskip} 
%\let\overbfile\verbfile
\def\verbfile#1{\newdisplay\verbatiminput{#1}\newdisplay}


%\newcommand{\isc}{Scientific computing\xspace}
\newcommand{\pdir}{practical directory\xspace} 
\newcommand{\bt}{Bioinformatics Toolbox\xspace} 
\newcommand{\isc}{Introduction to Scientific Computing\xspace} 
\renewcommand{\ca}{Cellular Automata}

\newcommand{\tx}[1]{\texttt{#1}}


\figcapbold

\graphicspath{
{../../syllabus/figures/figs_sequence_alignment/}
}
%%%%%%%%%%%%%%%%%%%%%%%%%%%%%%%%%%%%%%%%%%%%%%%%%%%%%%%%%%%%%%%%%%%%%%%%

\begin{document}
\section*{Practical Example: Hamming distance}

\begin{Exercise}[title={Hamming distance of two DNA sequences (10)}]

  In this assignment you will implement an algorithm in Matlab for
  computing the Hamming distance of two DNA sequences and displaying
  the matching positions.

\begin{enumerate}[leftmargin=*]

\item Study the file \vb|hamming.m| in the subdirectory
  \vb|input|. This is a so-called Matlab M-file. The appendix on
  page~\pageref{page:M-skelet} gives a listing of this M-file. Study
  the code and the comments, these should be self-evident. If
  necessary, consult the Matlab tutorial of the first practical, or
  the \vb|help| function of Matlab itself.

\item Load the file \vb|hamming.m| in an editor, and add the missing
  code. The program should compute a number HD that equals the Hamming
  distance of two strings \vb|s| and \vb|t|, and print this number HD
  on the screen. Call your program \vb|hamming1.m|.

\item Run your program \vb|hamming1.m| in Matlab. It will use the file
  \vb|input.txt| as input. Check that your program gives the correct
  answer for the strings in the input file.

\item Extend your program so that it displays the input strings below
  one another, with a line in between which has a '\vb?|?'  symbol on
  position \vb|k| if a match occurs between the input strings at that
  position, and a space otherwise, like in the following example:
\begin{center}
\small
    \begin{tabular}{cccc}
      C&  A&  G& G \\
    \vc&\vc&   & \vc\\      
      C&  A&  T& G \\
   \end{tabular}
\end{center}
Call your program \vb|hamming2.m|.

\item Run your program \vb|hamming2.m|, which again has \vb|input.txt|
  as input. Check that your program displays the correct answer on 
  the screen for the strings in the input file.

\item Instead of printing the results on the screen by the \vb|disp|
  or \vb|printf| function you can also print the results to a file by
  the \vb|fprintf| function, as follows. Include the following
  lines\footnote{These can also be found in the file
    \tx{print\_to\_file.m} in the subdirectory \tx{input}.}  at the
  end of your program \vb|hamming2.m|, and add the missing code for
  printing. The code for printing the value of the Hamming distance
  has already been inserted. Consult Matlab's \vb|help| function about
  printing (strings) of characters.

\verbatiminput{input/print_to_file.m} 

Call your extended program \vb|hamming3.m|. Run it again and check
that the output file \vb|hamming3-output.txt| contains the correct
information.

\end{enumerate}
\end{Exercise}

\textbf{Hand in}: 
\begin{itemize}
\item a concise report in PDF format (preferably generated by LaTeX),
  in which:
\begin{enumerate}[leftmargin=*, label=\textbf{\alph*.}, noitemsep] 
\item you describe, for each part of every assignment, how you arrived
  at the solution,
\item you answer all the questions posed in the assignments.
\end{enumerate}
Include the file \vb|hamming3-output.txt| also in your report. In
LaTeX this can easily be done by using the command
\vb|\verbatiminput{hamming3-output.txt}| (you need to include
\vb|\usepackage{verbatim}| in the preamble of your LaTeX document).

Write at the top of the first page of your report: ``\textbf{\isc,
  Practical Example}'', followed by your name(s), student number(s), and the date
when you hand in the report.
  
\item the files \vb|hamming1.m|, \vb|hamming2.m|, \vb|hamming3.m|
  containing your final implementations.

\item The file \vb|hamming3-output.txt| containing the Hamming
  distance and alignment of the input strings. 

\end{itemize}
All files have to be handed in as \textbf{a single archive}
(called \vb|YourName.zip| or \vb|YourName.tgz|), where ``YourName'' is
the concatenation of your last names (or your last name, if you work
individually). See Nestor for the address to which you have to send
the archive.



\newpage
\section*{Appendix: the skeleton program \texttt{hamming.m}}
\label{page:M-skelet}
{\small \verbatiminput{input/hamming.m}}

\end{document}
% 
LocalWords:  Matlab PracticalExample HD LaTeX
